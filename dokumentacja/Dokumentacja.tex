\documentclass[a4paper, 11pt]{article}
\usepackage[polish]{babel}
\usepackage[MeX]{polski}
\usepackage[utf8]{inputenc}
\usepackage[T1]{fontenc}
%\usepackage{times}
\usepackage{graphicx,wrapfig}
%\usepackage{anysize}
%\usepackage{tikz}
%\usetikzlibrary{calc,through,backgrounds,positioning}
\usepackage{anysize}
\usepackage{float}
%\usepackage{stmaryrd}
%\usepackage{amssymb}
%\usepackage{amsthm}
%\marginsize{3cm}{3cm}{3cm}{3cm}
%\usepackage{amsmath}
%\usepackage{color}
%\usepackage{listings}
%\usepackage{enumerate}
%\lstloadlanguages{Ada,C++}


\begin{document}	
	% \noindent -  w tym akapicie nie bedzie wciecia
	% \ indent - to jest aut., ale powoduje ze jest wciecie
	% \begin{flushleft}, flushright, center - wyrownianie akapitu
	% \textbf{pogrubiany tekst}
	% \textit{kursywa} 
	% 					STRONY 
	%  http://www.codecogs.com/latex/eqneditor.php 
	%  http://www.matematyka.pl/latex.htm
	% 
	\begin{titlepage}
		
		
		
		\newcommand{\HRule}{\rule{\linewidth}{0.5mm}} % Defines a new command for the horizontal lines, change thickness here
		
		\center % Center everything on the page
		
		%----------------------------------------------------------------------------------------
		%	HEADING SECTIONS
		%----------------------------------------------------------------------------------------
		
		\textsc{\LARGE Akademia Górniczo-Hutnicza im. Stanisława Staszica}\\[1.5cm] % Name of your university/college
		\textsc{\Large Kraków}\\[0.5cm] % Major heading such as course name
		\textsc{\large }\\[0.5cm] % Minor heading such as course title
		
		%----------------------------------------------------------------------------------------
		%	TITLE SECTION
		%----------------------------------------------------------------------------------------
		
		\HRule \\[0.4cm]
		{\fontsize{38}{50}\selectfont Generator specyfikacji logicznej}
		%	{ \Huge \bfseries} Symulator pożaru lasu\\[0.3cm] % Title of your document
		\HRule \\[5.5cm]
		
		%----------------------------------------------------------------------------------------
		%	AUTHOR SECTION
		%----------------------------------------------------------------------------------------
	
\begin{minipage}{0.4\textwidth}
\begin{flushleft} \large 
\emph{Autorzy:}\\
Marcin \textsc{Jędrzejczyk}\\ % Your name
Paweł \textsc{Ogorzały} \\

\end{flushleft}
\end{minipage}
~
\begin{minipage}{0.4\textwidth}
\begin{flushright} \large
\emph{Prowadzący:}\\
 Dr inż. Radosław \textsc{Klimek}  % Supervisor's Name
\end{flushright}
\end{minipage} \\[5cm]
		
		
		%----------------------------------------------------------------------------------------
		%	DATE SECTION
		%----------------------------------------------------------------------------------------
		
		{\large \today}\\[3cm] % Date, change the \today to a set date if you want to be precise
		
		%----------------------------------------------------------------------------------------
		%	LOGO SECTION
		%----------------------------------------------------------------------------------------
		
		%\includegraphics{Logo}\\[1cm] % Include a department/university logo - this will require the graphicx package
		
		%----------------------------------------------------------------------------------------
		
		\vfill % Fill the rest of the page with whitespace
		
	\end{titlepage}
	

	
	%\tableofcontents
	\vfill
	\newpage
	%\pagebreak
	
	
	
	%\setlength{\parskip}{1ex plus 0.5ex minus 0.2ex}
	
	\section{Wstęp}
    %	
	
	\indent
	\section{Cel projektu}
	%w sumie opisać co mamy zrobić	
	Celem projektu jest wytworzenie programu, który dla podanego diagramu będzie w stanie go sparsować do formatu pozwalającego na wygenerowanie specyfikacji logicznej.
	
	\section{Powód tworzenia generatora}
	%Klimek sam wyisał czemu, musimy to tu wstawić
	\begin{itemize}
	\item Ręczne tworzenie specyfikacji logiki jest trudne dla niedoświadczonych w tym użytkowników.
	\item Formalna weryfikacja modelu oprogramowania pozwala obniżyć koszty i zwiększyć niezawodność.
	\item Brak takich narzędzi.
	\end{itemize}
	\section{Ważne}
	\begin{itemize}
	\item Diagram aktywności musi składać się z wcześniej zdefiniowanych wzorców, zagnieżdżanie jest dozwolone.
	\item Diagram aktywności składa się tylko z atomicznych aktywności, zidentyfikowanych podczas tworzenia scenariuszy przypadków użycia.
	\item Generator musi działać automatycznie, usuwa to błąd ludzki.
	\end{itemize}
	\section{Algorytmy}
	
	Przykładowe wzorce:
	\begin{itemize}
	\item Sekwencja, sequence
	\item Współbieżność, concurrent fork/join
	\item Pętla while, loop while
	\item Rozgałęzienie, branching
	\end{itemize}
		
	\noindent Wyrażenie logiczne $W_L$ jest strukturą stworzoną według pniższych zasad:\\
	\begin{itemize}
	\item każdy elementarny zbiór $pat(a_i)$, gdzie $i>0$ i każde $a_i$ jest formułą atomiczną, jest wyrażeniem logicznym,
	\item każde $pat(A)$, gdzie $i>0$ i każde $A_i$ jest albo
	\begin{itemize}
	\item atomiczną formułą lub
	\item logicznym wyrażeniem $pat()$
	\end{itemize}
	także jest wyrażeniem logicznym.
	\end{itemize}
	
	\noindent Wstępny algorytm:
	\begin{enumerate}
	\item Analiza diagramów aktywności w celu wyciągnięcia z nich wcześniej zdefiniowanych wzorców przepływu.
	\item Przetłumaczenie wyłuskanych wzorców na wyrażenia logiczne $ W_L$.
	\item Generowanie specyfikacji logicznej $L$ z wyrażeń logicznych, %i.e. receiving a set of temporal logics formulas.
	\end{enumerate}
		\noindent Algorytm generujący specyfikację logiczną:\\
		\begin{enumerate}
		\item Na początku specyfikacja jest pusta, np. L={\o};
		\item Najbardziej zagnieżdżone wzorce są przetwarzane jako pierwsze, a następnie mniej zagnieżdżone;
		\item Jeśli obecnie analizowany wzorzec składa się wyłącznie z formuł atomicznych, specyfikacja logiczna jest rozszerzana, poprzez sumowanie zbiorów, których formuły są złączone z obecnie analizowanym wzorcem pat(), np. $L=L \cup pat() $;
		\item Jeżeli jakiś argument jest wzorem sam w sobie to:
		\begin{itemize}
		\item po pierwsze formuła $f1$ ?? , a potem
		\item formuła $f4$ ??
		\end{itemize}
		tego wzoru(jeśli jakiegoś), lub w innym wypadku wziąć pod uwagę tylko najbardziej zagnieżdżony daleko? na lewo lub prawo,odpowiednio, są podstawiane osobno w miejsce wzorca jako argument.
		\end{enumerate}
	\section{Technologie}
	Wydaję mi się, że języki funkcyjne mogłyby się tu dobrze sprawdzić, może Erlang.\\
	Chociaż w Javie mogłoby się to łatwiej napisać(lepie znamy Jave niż Erlanga).  
	
	
	\section{Literatura}
	%\indent	
	\textbf{Radosław Klimek:} From Extraction of Logical Specifications to Deduction-Based Formal Verification of Requitements Models. Strony 61-75.\\
	
	
	
	
\end{document}
